My goals for this course are to learn how to conduct computer science research ethically and effectively. Computer science research, in particular, seems to skirt traditional institutional review board (IRB) review with measuring or monitoring security and privacy online. Navigating IRB can be tricky and failure to adhere to federal guidelines and procedures can result in withdrawing or retracting research from refereed venues. Learning how computer science approaches human factors can be directly applied to my research interests and my job. More specifically, my success as a computer scientist hinges on how well I communicate research efforts in formal avenues and I hope to learn tips and tricks I can use to make research writing easier. In the short term I hope to use what I learn from this class to complete program requirements for a PhD in Security.

I currently work as a data scientist for the Nebraska Applied Research Institute and earned undergraduate and masters degrees in Computer Engineering from the University of Central Florida. In previous lives I was a USN nuke, VA photographer, NCCDC winner, Hack@UCF mom, and darknet marketplace miner. My current research centers on the application of machine learning to intrusion detection. My current research profile on Google Scholar matches the photo shown in Figure \ref{fig:profile}.

\begin{figure}[h!]
   \centering
    \includegraphics[width=.35\textwidth]{lawrence.png}
    \caption{Profile}
    \label{fig:profile}
\end{figure}

\subsection{Question and Answer - Lawrence}
\subsubsection{Question \#1}
Keith here -- With the research in computer security and privacy, what do you think the differences in pure research (done under an IRB) and "product improvement" done under a user agreement?

Answer - Many companies seem to have their own internal IRB-esque process for product improvement. PayPal, for example, had our project manager advertise and find users for studies. As companies don't seem to be attached to universities or medical centers requiring stringent IRB adherence, there seems to be a gap that could be abused. Provided that the company follows the guidelines for IRB there is little different between product-guided research using human subjects and academic-guided research using human subjects save for who gets the intellectual property in the end.

\subsubsection{Question \#2}
Prajjwal here, hi -- What type of (network, side-channel, physical, or something else) intrusion detection are
you researching?

I'm studying network intrusion detection.
